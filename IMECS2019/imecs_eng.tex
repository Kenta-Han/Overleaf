
%% IAENG_pub.tex 2010/08/30
%% It is based on the bare_jrnl.tex V1.3 2007/01/11 by Michael Shell
%% see http://www.michaelshell.org/
%% for current contact information.
%%
%% This is a skeleton file demonstrating the use of IAENGtran.cls
%% (requires IAENGtran.cls version 1.7 or later) with an IAENG journal/conference paper.
%%
%% Support sites:
%% http://www.michaelshell.org/tex
%% http://www.ctan.org/tex-archive/macros/latex/contrib/IEEEtran/

% *** Authors should verify (and, if needed, correct) their LaTeX system  ***
% *** with the testflow diagnostic prior to trusting their LaTeX platform ***
% *** with production work. IAENG's font choices can trigger bugs that do  ***
% *** not appear when using other class files.                            ***
% The testflow support page is at:
% http://www.michaelshell.org/tex/testflow/


%%*************************************************************************
%% Legal Notice:
%% This code is offered as-is without any warranty either expressed or
%% implied; without even the implied warranty of MERCHANTABILITY or
%% FITNESS FOR A PARTICULAR PURPOSE!
%% User assumes all risk.
%% In no event shall IAENG or any contributor to this code be liable for
%% any damages or losses, including, but not limited to, incidental,
%% consequential, or any other damages, resulting from the use or misuse
%% of any information contained here.
%%
%% All comments are the opinions of their respective authors and are not
%% necessarily endorsed by the IAENG.
%%
%% This work is distributed under the LaTeX Project Public License (LPPL)
%% ( http://www.latex-project.org/ ) version 1.3, and may be freely used,
%% distributed and modified. A copy of the LPPL, version 1.3, is included
%% in the base LaTeX documentation of all distributions of LaTeX released
%% 2003/12/01 or later.
%% Retain all contribution notices and credits.
%% ** Modified files should be clearly indicated as such, including  **
%% ** renaming them and changing author support contact information. **
%%
%%*************************************************************************

% Note that the a4paper option is mainly intended so that authors in
% countries using A4 can easily print to A4 and see how their papers will
% look in print - the typesetting of the document will not typically be
% affected with changes in paper size (but the bottom and side margins will).
% Use the testflow package mentioned above to verify correct handling of
% both paper sizes by the user's LaTeX system.
%
% Also note that the "draftcls" or "draftclsnofoot", not "draft", option
% should be used if it is desired that the figures are to be displayed in
% draft mode.
%
\documentclass[journal]{IAENGtran}
%
% If IAENGtran.cls has not been installed into the LaTeX system files,
% manually specify the path to it like:
% \documentclass[journal]{../sty/IAENGtran}





% Some very useful LaTeX packages include:
% (uncomment the ones you want to load)


% *** MISC UTILITY PACKAGES ***
%
%\usepackage{ifpdf}
% Heiko Oberdiek's ifpdf.sty is very useful if you need conditional
% compilation based on whether the output is pdf or dvi.
% usage:
% \ifpdf
%   % pdf code
% \else
%   % dvi code
% \fi
% The latest version of ifpdf.sty can be obtained from:
% http://www.ctan.org/tex-archive/macros/latex/contrib/oberdiek/
% Also, note that IAENGtran.cls V1.7 and later provides a builtin
% \ifCLASSINFOpdf conditional that works the same way.
% When switching from latex to pdflatex and vice-versa, the compiler may
% have to be run twice to clear warning/error messages.






% *** CITATION PACKAGES ***
%
%\usepackage{cite}
% cite.sty was written by Donald Arseneau
% V1.6 and later of IAENGtran pre-defines the format of the cite.sty package
% \cite{} output to follow that of IAENG. Loading the cite package will
% result in citation numbers being automatically sorted and properly
% "compressed/ranged". e.g., [1], [9], [2], [7], [5], [6] without using
% cite.sty will become [1], [2], [5]--[7], [9] using cite.sty. cite.sty's
% \cite will automatically add leading space, if needed. Use cite.sty's
% noadjust option (cite.sty V3.8 and later) if you want to turn this off.
% cite.sty is already installed on most LaTeX systems. Be sure and use
% version 4.0 (2003-05-27) and later if using hyperref.sty. cite.sty does
% not currently provide for hyperlinked citations.
% The latest version can be obtained at:
% http://www.ctan.org/tex-archive/macros/latex/contrib/cite/
% The documentation is contained in the cite.sty file itself.






% *** GRAPHICS RELATED PACKAGES ***
%
\ifCLASSINFOpdf
   \usepackage[pdftex]{graphicx}
  % declare the path(s) where your graphic files are
  % \graphicspath{{../pdf/}{../jpeg/}}
  % and their extensions so you won't have to specify these with
  % every instance of \includegraphics
   \DeclareGraphicsExtensions{.pdf,.jpeg,.png}
\else
  % or other class option (dvipsone, dvipdf, if not using dvips). graphicx
  % will default to the driver specified in the system graphics.cfg if no
  % driver is specified.
   \usepackage[dvips]{graphicx}
  % declare the path(s) where your graphic files are
  % \graphicspath{{../eps/}}
  % and their extensions so you won't have to specify these with
  % every instance of \includegraphics
   \DeclareGraphicsExtensions{.eps}
\fi
% graphicx was written by David Carlisle and Sebastian Rahtz. It is
% required if you want graphics, photos, etc. graphicx.sty is already
% installed on most LaTeX systems. The latest version and documentation can
% be obtained at:
% http://www.ctan.org/tex-archive/macros/latex/required/graphics/
% Another good source of documentation is "Using Imported Graphics in
% LaTeX2e" by Keith Reckdahl which can be found as epslatex.ps or
% epslatex.pdf at: http://www.ctan.org/tex-archive/info/
%
% latex, and pdflatex in dvi mode, support graphics in encapsulated
% postscript (.eps) format. pdflatex in pdf mode supports graphics
% in .pdf, .jpeg, .png and .mps (metapost) formats. Users should ensure
% that all non-photo figures use a vector format (.eps, .pdf, .mps) and
% not a bitmapped formats (.jpeg, .png). IAENG frowns on bitmapped formats
% which can result in "jaggedy"/blurry rendering of lines and letters as
% well as large increases in file sizes.
%
% You can find documentation about the pdfTeX application at:
% http://www.tug.org/applications/pdftex





% *** MATH PACKAGES ***
%
%\usepackage[cmex10]{amsmath}
% A popular package from the American Mathematical Society that provides
% many useful and powerful commands for dealing with mathematics. If using
% it, be sure to load this package with the cmex10 option to ensure that
% only type 1 fonts will utilized at all point sizes. Without this option,
% it is possible that some math symbols, particularly those within
% footnotes, will be rendered in bitmap form which will result in a
% document that can not be IAENG compliant!
%
% Also, note that the amsmath package sets \interdisplaylinepenalty to 10000
% thus preventing page breaks from occurring within multiline equations. Use:
%\interdisplaylinepenalty=2500
% after loading amsmath to restore such page breaks as IAENGtran.cls normally
% does. amsmath.sty is already installed on most LaTeX systems. The latest
% version and documentation can be obtained at:
% http://www.ctan.org/tex-archive/macros/latex/required/amslatex/math/





% *** SPECIALIZED LIST PACKAGES ***
%
%\usepackage{algorithmic}
% algorithmic.sty was written by Peter Williams and Rogerio Brito.
% This package provides an algorithmic environment for describing algorithms.
% You can use the algorithmic environment in-text or within a figure
% environment to provide for a floating algorithm. Do NOT use the algorithm
% floating environment provided by algorithm.sty (by the same authors) or
% algorithm2e.sty (by Christophe Fiorio) as IAENG does not use dedicated
% algorithm float types and packages that provide these will not provide
% correct IAENG style captions. The latest version and documentation of
% algorithmic.sty can be obtained at:
% http://www.ctan.org/tex-archive/macros/latex/contrib/algorithms/
% There is also a support site at:
% http://algorithms.berlios.de/index.html
% Also of interest may be the (relatively newer and more customizable)
% algorithmicx.sty package by Szasz Janos:
% http://www.ctan.org/tex-archive/macros/latex/contrib/algorithmicx/




% *** ALIGNMENT PACKAGES ***
%
%\usepackage{array}
% Frank Mittelbach's and David Carlisle's array.sty patches and improves
% the standard LaTeX2e array and tabular environments to provide better
% appearance and additional user controls. As the default LaTeX2e table
% generation code is lacking to the point of almost being broken with
% respect to the quality of the end results, all users are strongly
% advised to use an enhanced (at the very least that provided by array.sty)
% set of table tools. array.sty is already installed on most systems. The
% latest version and documentation can be obtained at:
% http://www.ctan.org/tex-archive/macros/latex/required/tools/


%\usepackage{mdwmath}
%\usepackage{mdwtab}
% Also highly recommended is Mark Wooding's extremely powerful MDW tools,
% especially mdwmath.sty and mdwtab.sty which are used to format equations
% and tables, respectively. The MDWtools set is already installed on most
% LaTeX systems. The lastest version and documentation is available at:
% http://www.ctan.org/tex-archive/macros/latex/contrib/mdwtools/


% IAENGtran contains the IAENGeqnarray family of commands that can be used to
% generate multiline equations as well as matrices, tables, etc., of high
% quality.


%\usepackage{eqparbox}
% Also of notable interest is Scott Pakin's eqparbox package for creating
% (automatically sized) equal width boxes - aka "natural width parboxes".
% Available at:
% http://www.ctan.org/tex-archive/macros/latex/contrib/eqparbox/





% *** SUBFIGURE PACKAGES ***
%\usepackage[tight,footnotesize]{subfigure}
% subfigure.sty was written by Steven Douglas Cochran. This package makes it
% easy to put subfigures in your figures. e.g., "Figure 1a and 1b". For IAENG
% work, it is a good idea to load it with the tight package option to reduce
% the amount of white space around the subfigures. subfigure.sty is already
% installed on most LaTeX systems. The latest version and documentation can
% be obtained at:
% http://www.ctan.org/tex-archive/obsolete/macros/latex/contrib/subfigure/
% subfigure.sty has been superceeded by subfig.sty.



%\usepackage[caption=false]{caption}
%\usepackage[font=footnotesize]{subfig}
% subfig.sty, also written by Steven Douglas Cochran, is the modern
% replacement for subfigure.sty. However, subfig.sty requires and
% automatically loads Axel Sommerfeldt's caption.sty which will override
% IAENGtran.cls handling of captions and this will result in nonIAENG style
% figure/table captions. To prevent this problem, be sure and preload
% caption.sty with its "caption=false" package option. This is will preserve
% IAENGtran.cls handing of captions. Version 1.3 (2005/06/28) and later
% (recommended due to many improvements over 1.2) of subfig.sty supports
% the caption=false option directly:
%\usepackage[caption=false,font=footnotesize]{subfig}
%
% The latest version and documentation can be obtained at:
% http://www.ctan.org/tex-archive/macros/latex/contrib/subfig/
% The latest version and documentation of caption.sty can be obtained at:
% http://www.ctan.org/tex-archive/macros/latex/contrib/caption/




% *** FLOAT PACKAGES ***
%
%\usepackage{fixltx2e}
% fixltx2e, the successor to the earlier fix2col.sty, was written by
% Frank Mittelbach and David Carlisle. This package corrects a few problems
% in the LaTeX2e kernel, the most notable of which is that in current
% LaTeX2e releases, the ordering of single and double column floats is not
% guaranteed to be preserved. Thus, an unpatched LaTeX2e can allow a
% single column figure to be placed prior to an earlier double column
% figure. The latest version and documentation can be found at:
% http://www.ctan.org/tex-archive/macros/latex/base/



%\usepackage{stfloats}
% stfloats.sty was written by Sigitas Tolusis. This package gives LaTeX2e
% the ability to do double column floats at the bottom of the page as well
% as the top. (e.g., "\begin{figure*}[!b]" is not normally possible in
% LaTeX2e). It also provides a command:
%\fnbelowfloat
% to enable the placement of footnotes below bottom floats (the standard
% LaTeX2e kernel puts them above bottom floats). This is an invasive package
% which rewrites many portions of the LaTeX2e float routines. It may not work
% with other packages that modify the LaTeX2e float routines. The latest
% version and documentation can be obtained at:
% http://www.ctan.org/tex-archive/macros/latex/contrib/sttools/
% Documentation is contained in the stfloats.sty comments as well as in the
% presfull.pdf file. Do not use the stfloats baselinefloat ability as IAENG
% does not allow \baselineskip to stretch. Authors submitting work to the
% IAENG should note that IAENG rarely uses double column equations and
% that authors should try to avoid such use. Do not be tempted to use the
% cuted.sty or midfloat.sty packages (also by Sigitas Tolusis) as IAENG does
% not format its papers in such ways.


%\ifCLASSOPTIONcaptionsoff
%  \usepackage[nomarkers]{endfloat}
% \let\MYoriglatexcaption\caption
% \renewcommand{\caption}[2][\relax]{\MYoriglatexcaption[#2]{#2}}
%\fi
% endfloat.sty was written by James Darrell McCauley and Jeff Goldberg.
% This package may be useful when used in conjunction with IAENGtran.cls'
% captionsoff option. Some IAENG journals/societies require that submissions
% have lists of figures/tables at the end of the paper and that
% figures/tables without any captions are placed on a page by themselves at
% the end of the document. If needed, the draftcls IAENGtran class option or
% \CLASSINPUTbaselinestretch interface can be used to increase the line
% spacing as well. Be sure and use the nomarkers option of endfloat to
% prevent endfloat from "marking" where the figures would have been placed
% in the text. The two hack lines of code above are a slight modification of
% that suggested by in the endfloat docs (section 8.3.1) to ensure that
% the full captions always appear in the list of figures/tables - even if
% the user used the short optional argument of \caption[]{}.
% IAENG papers do not typically make use of \caption[]'s optional argument,
% so this should not be an issue. A similar trick can be used to disable
% captions of packages such as subfig.sty that lack options to turn off
% the subcaptions:
% For subfig.sty:
% \let\MYorigsubfloat\subfloat
% \renewcommand{\subfloat}[2][\relax]{\MYorigsubfloat[]{#2}}
% For subfigure.sty:
% \let\MYorigsubfigure\subfigure
% \renewcommand{\subfigure}[2][\relax]{\MYorigsubfigure[]{#2}}
% However, the above trick will not work if both optional arguments of
% the \subfloat/subfig command are used. Furthermore, there needs to be a
% description of each subfigure *somewhere* and endfloat does not add
% subfigure captions to its list of figures. Thus, the best approach is to
% avoid the use of subfigure captions (many IAENG journals avoid them anyway)
% and instead reference/explain all the subfigures within the main caption.
% The latest version of endfloat.sty and its documentation can obtained at:
% http://www.ctan.org/tex-archive/macros/latex/contrib/endfloat/
%
% The IAENGtran \ifCLASSOPTIONcaptionsoff conditional can also be used
% later in the document, say, to conditionally put the References on a
% page by themselves.





% *** PDF, URL AND HYPERLINK PACKAGES ***
%
%\usepackage{url}
% url.sty was written by Donald Arseneau. It provides better support for
% handling and breaking URLs. url.sty is already installed on most LaTeX
% systems. The latest version can be obtained at:
% http://www.ctan.org/tex-archive/macros/latex/contrib/misc/
% Read the url.sty source comments for usage information. Basically,
% \url{my_url_here}.





% *** Do not adjust lengths that control margins, column widths, etc. ***
% *** Do not use packages that alter fonts (such as pslatex).         ***
% There should be no need to do such things with IAENGtran.cls V1.6 and later.
% (Unless specifically asked to do so by the journal or conference you plan
% to submit to, of course. )


% correct bad hyphenation here
% \hyphenation{op-tical net-works semi-conduc-tor}


\begin{document}
%
% paper title
% can use linebreaks \\ within to get better formatting as desired
\title{Analogical Information Presentation Method Based on Already Visited Spot for Understanding of Unvisited Area}
%
%
% author names and IAENG memberships
% note positions of commas and nonbreaking spaces ( ~ ) LaTeX will not break
% a structure at a ~ so this keeps an author's name from being broken across
% two lines.
% use \thanks{} to gain access to the first footnote area
% a separate \thanks must be used for each paragraph as LaTeX2e's \thanks
% was not built to handle multiple paragraphs
%

\author{Michael~Shell,~\IAENGmembership{Member,~IAENG,}
        John~Doe,~\IAENGmembership{Senior Member,~IAENG,}
        and~Jane~Doe,~\IAENGmembership{Fellow,~IAENG}% <-this % stops a space
\thanks{Manuscript received April XX, 20XX; revised June XX, 20XX. (Write the date on
which you submitted your paper for review.) This work was supported
in part by the U.S. Department of Commerce under Grant BS123456
(sponsor and financial support acknowledgment goes here). Paper
titles should be written in uppercase and lowercase letters, not all
uppercase. Avoid writing long formulas with subscripts in the title;
short formulas that identify the elements are fine.}
\thanks{M. Shell is with the Department
of Electrical and Computer Engineering, Georgia Institute of
Technology, Atlanta,
GA, 30332 USA e-mail: (see http://www.michaelshell.org/contact.html).}% <-this % stops a space
\thanks{J. Doe and J. Doe are with Anonymous University.}}% <-this % stops a space


% make the title area
\maketitle

\pagestyle{empty}
\thispagestyle{empty}

%%%%%%%%%%%%%%%%%%%%%%%%%%%%%%%%%%%%%%%%%%
%%%%%%%%%%%%%%%%%%%%%%%%%%%%%%%%%%%%%%%%%%
\begin{abstract}
%\boldmath


% 近年,観光スポットを決める時にWeb上の観光情報を活用して計画を立てることが多くなっている.
In recent years, when planning tourist spots, planning is often made by utilizing tourist information on the Web.
% しかし,ユーザが多くのエリアから訪問したいエリアを決めた上で,さらに自分のイメージに合う観光スポットを探すのは膨大な時間と労力を必要とする.
However, after deciding the area you want to visit from many areas, the user also needs enormous amount of time and effort to find tourist spots that match your image.
% また,ユーザが未訪問スポットに対して期待と不安を感じる場合がある.
In addition, there are cases where the user feels expectation and anxiety with respect to the unvisited spot.
% 本研究では,ユーザの未知なスポットに対する理解を支援するためには,既に訪問したことがある観光スポットの特徴を未訪問スポットにあてはめて理解を支援する類推情報提示を提案する.
In this research, in order to support understanding of users' unknown spots, we propose analogy information presentation that supports the understanding by fitting the features of tourist spots that have already visited to unvisited spots.
% 観光スポット自身の特徴を重視するため,各観光スポットの特徴抽出に,ユーザが入力した観光スポットのすべてのレビュー,対象エリアの観光スポットのすべてのレビューを使用する.
In order to emphasize the features of the tourist spots themselves, extraction of features of each tourist spot is done by work using all reviews of tourist spots entered by the user, all reviews of tourist spots in the target area.
% また,プロトタイプシステムを構築し,既訪問スポットと未訪問スポットとの類推情報の効果を検証する評価実験を行う.
We also conduct an evaluation experiment to construct the prototype system and verify the effect of the analogy information between the visited spot and the unvisited spot.
\end{abstract}

%%%%%%%%%%%%%%%%%%%%%%%%%%%%%%%%%%%%%%%%%%
%%%%%%%%%%%%%%%%%%%%%%%%%%%%%%%%%%%%%%%%%%
\begin{IAENGkeywords}
% 観光スポット,類推,理解支援,レビュー,コサイン類似度,TFIDF,調和平均
tourist spots, analogy, understanding support, reviews, cosine similarity, TFIDF, harmonic mean.
\end{IAENGkeywords}

\IAENGpeerreviewmaketitle


%%%%%%%%%%%%%%%%%%%%%%%%%%%%%%%%%%%%%%%%%%
%%%%%%%%%%%%%%%%%%%%%%%%%%%%%%%%%%%%%%%%%%
\section{Introduction}
\label{sec:Introduction}
%%%%%%%%%%%%%%%%%%%%%%%%%%%%%%%%%%%%%%%%%%
%%%%%%%%%%%%%%%%%%%%%%%%%%%%%%%%%%%%%%%%%%
% 旅行先を決定する時,旅行者は観光スポット検索サイトや観光情報に関連する書籍を見て観光スポットを選び,旅行計画を立てる.
\IAENGPARstart{W}{hen} deciding the travel destination, the traveler selects tourist spots by planning a travel plan, watching tourist spots search sites and books related to tourist information.
% しかし,ユーザにとって訪問したいエリアを決定した後,さらにエリア内に数多く存在する観光スポットから,自身のイメージから外れない観光スポットを見つけることは容易ではない.
However, after deciding the area you want to visit from many areas,and further from their many tourist spots in the area is not easy to find.
% 行きたい観光スポットが決まっていない場合ではランキングやおすすめ情報を見て観光スポットを決めることが多くなると考えられる.
In the case where the tourist spots desired to go are not decided, it is considered that it is more likely to decide tourist spots by looking at ranking and recommendation information.
% この時,ユーザが選択した観光スポットに対するイメージが曖昧になるため不安を感じる場合がある.
At this time, the image for the tourist spots selected by the user becomes ambiguous, which may cause anxiety.


% 近年,観光業とソーシャルネットワーキングサービスの発展スピードが加速しており,体験した観光スポットに対するレビューを観光スポット検索サイトに投稿しているユーザが増加している.
In recent years, the speed of development of tourism industry and social networking service is accelerating, and the number of users who post reviews on tourist spots experienced to the tourist spot search site is increasing.
% さまざまな観光スポットを効果的に理解するためには,既存の情報をもとにして,未知な情報と既知な情報との対応関係を考えることが不可欠となる.
In order to effectively understand various tourist spots, it is essential to consider the correspondence between unknown information and existing information based on existing information.
% この考え方は,以前に経験した事柄(ベースと呼ぶ)を,現在直面している事柄あるいは問題(ターゲットと呼ぶ)にあてはめる類推に相当する.
This way of thinking is equivalent to analogy which applies to the things by previous experiences (called "bases"), or problems (called "targets").
% たとえば,金沢の「にし茶屋街」という未知なスポットに対して既訪問の京都の「花見小路」と似ていると説明するとイメージの理解がしやすくすることがある.
For example, whereas unknown spots such as "Kanazawa's Nisityayagai", if you explain that it is similar to the already visited "Kyoto Hanamikoji", it may make it easier to understand the image.

% 本研究では,ユーザの未知なスポットに対する理解を支援するため,既に訪問したことがある観光スポットの特徴を未訪問スポットにあてはめて理解を支援する類推情報提示を提案する.
In this research, in order to support understanding of users' unknown spots, we propose analogy information presentation that supports the understanding by fitting the features of tourist spots that have already visited to unvisited spots.
% 具体的には,ユーザが入力した既訪問スポットと未訪問エリアから,レビュを用いて既訪問スポット内の各スポットの独特な特徴と未訪問エリア内の各スポットの独特な特徴を抽出し,比較を行って類推情報を提示する.
Specifically, from the already visited spot and the unvisited area entered by the user, we use the review to extract the unique features of each spot in the already visited spot and the unique features of each spot in the unvisited area, compare and present analogy information.
% このプロトタイプシステムにより,ユーザが未訪問エリアに対する理解の支援を目指す.
With this prototype system, users aim to support understanding of unvisited areas.
% 図\ref{fig:Photo_Image}は提案手法の概念図である.
Fig. \ref{fig:Photo_Image} is a conceptual diagram of the proposed method.

% 本論文の構成は下記のとおりである.2節では関連研究について述べる.3節では提案手法の概要について述べる.4節では構築したプロトタイプシステムの効果を検証する評価実験と考察について述べる.最後に5節ではまとめと今後の課題について述べる.
The structure of this paper is as follows.
Section \ref{sec:Related Work} describes related research.
Section \ref{sec:Analogical Information Presentation Method} gives an overview of the proposed method.
Section \ref{sec:Preliminary Experiment} describes evaluation experiments and considerations to verify the effect of the constructed prototype system.
Section \ref{sec:Conclusions and Future Work} describes with conclusions and future work.

\begin{figure}[t]
  \begin{center}
    \includegraphics[clip,width=7.5cm,bb=0 0 720 540]{picture/Photo_Image_eng.png}
    % 未訪問エリアの理解支援のための既訪問スポットに基づく類推情報提示手法
    \caption{Analogical information presentation method based on already visited spot for understanding of unvisited area}
    \label{fig:Photo_Image}
   \end{center}
\end{figure}


%%%%%%%%%%%%%%%%%%%%%%%%%%%%%%%%%%%%%%%%%%
%%%%%%%%%%%%%%%%%%%%%%%%%%%%%%%%%%%%%%%%%%
\section{Related Work}
\label{sec:Related Work}
%%%%%%%%%%%%%%%%%%%%%%%%%%%%%%%%%%%%%%%%%%
%%%%%%%%%%%%%%%%%%%%%%%%%%%%%%%%%%%%%%%%%%


%%%%%%%%%%%%%%%%%%%%%%%%%%%%%%%%%%%%%%%%%%
%%%%%%%%%%%%%%%%%%%%%%%%%%%%%%%%%%%%%%%%%%
\section{Analogical Information Presentation Method}
\label{sec:Analogical Information Presentation Method}
%%%%%%%%%%%%%%%%%%%%%%%%%%%%%%%%%%%%%%%%%%
%%%%%%%%%%%%%%%%%%%%%%%%%%%%%%%%%%%%%%%%%%
% 我々は,未訪問エリアの理解支援のための既訪問スポットに基づく類推情報提示手法を提案する.
We propose an analogy information presentation method based on an already visited spot for supporting understanding of unvisited areas.
% 具体的にはまず,ユーザが既訪問の複数個の観光スポットと訪問したい観光スポットエリア情報を入力する.
Specifically, first, the user inputs a plurality of tourist spots that have been visited and tourist spot area information that user wishes to visit.
% 既訪問スポットレビューベクトルを使って既訪問スポット毎の特徴ベクトルを求める.
Use the already visited spot review vector to find the feature vector for each visited spot.
% 未訪問スポットも同様にエリア内の各スポットの特徴ベクトルを求める.
Similarly, the feature vector of each spot in the area is obtained for an unvisited spot.
% 次に,既訪問スポットレビューベクトルと未訪問スポットレビューベクトルの差分特徴に類似する特徴を持つ未既訪問観光スポット関連付けを行う.
Next, we associate unvisited tourist spots with features similar to the difference features between the visited spot review vector and the unvisited spot review vector.
% 最後に,TFIDFを用いて未訪問エリアの理解支援のための類推情報を定義し,ユーザに提示する.
Finally, analogy information for supporting understanding of unvisited areas is defined using TFIDF and presented to the user.

%%%%%%%%%%%%%%%%%%%%%%%%%%%%%%%%%%%%%%%%%%
\subsection{Relative features of tourist spots}
\label{subsec:Relative features of tourist spots}

\begin{figure}[t]
  \begin{center}
    \includegraphics[clip,width=7.5cm,bb=0 0 720 540]{picture/Photo_CosSim_eng.png}
    % 類似度計算概念図
    \caption{Concept of similarity calculation}
    \label{fig:Photo_CosSim}
    \end{center}
\end{figure}

% 本研究では,観光スポットの特徴は相対的な特徴を利用する.
In this research, features of tourist spots make use of relative features.
% 相対的な特徴とは,特定の観光スポットが,ある観光スポット集合に含まれた他の観光スポットと比較した場合における独特な特徴である.
The relative feature is a unique feature when a specific tourist spot is compared with other tourist spots included in a set of tourist spots.
% 例として,観光スポット集合内に鹿苑寺と清水寺が存在する場合を考える.
As an example, consider the case where there are "Rokuonji" and "Kiyomizudera" in the tourist spot group.
% このとき鹿苑寺の特徴は,金色,金箔,輝き等となり,清水寺の特徴は,舞台,胎内,一望等となる.
At this time, the features of "Rokuonji" will be gold color, gold leaf, glow, etc., the features of "Kiyomizudera" are the stage, the womb inside, the panoramic view etc.
% どちらも京都に存在する寺院であるため,京都や寺院に関連する特徴は独特な特徴として現れることがない.
Because both are temples existing in Kyoto, features related to Kyoto and temples do not appear as unique features.
% 次に,観光スポット集合内に東京都庁舎展望台と鹿苑寺が存在する場合を考える.
Next, consider the case where "the Tokyo Metropolitan Government Building Observatories" and "Rokuonji" exist within the tourist spot group.
% このとき鹿苑寺の特徴は,金閣寺,お寺,金色,京都等となり,東京都庁舎の特徴は,展望,夜景,新宿等となる.
Features of "Rokuonji" at this time will be Kinkakuji, a temple, golden color, Kyoto etc.
% 観光スポットのカテゴリーが大きく異なる場合であれば,カテゴリーとしての特徴が現れる.
Features of "the Tokyo Metropolitan Government Building Observatories" will be perspectives, night view, Shinjuku etc.
% また,スポット自身の特徴を表すことができる.
If the categories of tourist spots are largely different, features as categories will appear.
% 本研究では,あるスポットが集合内の他のスポットと比較するとき,より各スポットの特徴を明らかにできる相対的な特徴に着目して研究を行う.
Also, it can show the features of the spot itself.
In this research, when a certain spot compares with other spots in the set, we focus on the relative features that make it possible to clarify the features of each spot.

%%%%%%%%%%%%%%%%%%%%%%%%%%%%%%%%%%%%%%%%%%
\subsection{Calculation of similarity by cosine similarity}
\label{subsec:Calculation of similarity by cosine similarity}
% 既訪問スポットや未訪問スポットのレビューベクトルは,形態素解析器「mecab-ipadic-NEologd」で分かち書き(原型)したレビューを利用して作成する.
Review vectors of previously visited spots and unvisited spots are created using a discriminated (original) review with the morphological analyzer "mecab-ipadic-NEologd"\footnote{https://github.com/neologd/mecab-ipadic-neologd/}.
% その後,Doc2VecのDistributed Bag-of-Wordsを利用して,各スポットの全レビューを使って300次元で作成したベクトルを使う.
After that, using Distributed\footnote{https://radimrehurek.com/gensim/models/doc2vec.html} Bag-of-Words of Doc2Vec, we use a vector created in 300 dimensions using all reviews of each spot.
% 本稿に置いて,レビューデータは2016年09月末までじゃらんから取得したものを用いる.
In this paper, we will use the review data obtained from "Jalan"\footnote{https://www.jalan.net/kankou/} until the end of September 2016.

% スポット差分ベクトルは式\ref{math:Vector difference}として定義される.
The spot differential vector is defined as formula \ref{math:Vector difference}.
% スポット差分ベクトルを求めるスポットを除いたスポット集合の各スポットのスポットベクトルの平均値を引いた値となる.
Is the value obtained by subtracting the average value of the spot vectors of the spots of the spot group excluding the spot for which the spot differential vector is found.
% $spot_{set} =\{s_1,s_2,\dots,s_i,\dots,s_n\}$は既訪問スポット集合や未訪問スポット集合となっている.
$spot_{set} =\{s_1,s_2,\dots,s_i,\dots,s_n\}$1 is an already visited spot set or an unvisited spot set.
% また,$s_i$は集合内のある観光スポットを示している.
$s_i$ is a tourist spot in the set.

\begin{equation}
  v_i=s_i-average(spot_{set}-s_i)
    \label{math:Vector difference}
\end{equation}

% 既訪問スポットの各特徴差分ベクトル$v_i$と未訪問スポットの各特徴差分ベクトル$v_j$から,既訪問スポットと未訪問スポット間の相対的な特徴の類似度(図\ref{fig:photo_cossim})を求める.
From the feature difference vector $v_i$ of the visited spot and the feature difference vector $v_j$ of the unvisited spot, the relative feature similarity between the visited spot and the unvisited spot (Fig. \ref{fig:Photo_CosSim} ).
% 類似度計算には,コサイン尺度(式\ref{math:CosSim})を用いる.
For the similarity calculation, use the cosine scale (formula \ref{math: CosSim}).

\begin{eqnarray}
cos(v_i,v_j)=\frac{v_{i1}v_{j1}+v_{i2}v_{j2}+\cdots+v_{in}v_{jn}}
{\sqrt{v^2_{i1}+\cdots+v^2_{in}}\times\sqrt{v^2_{j1}+\cdots+v^2_{jn}}}
\label{math:CosSim}
\end{eqnarray}

% 既訪問スポットの各特徴ベクトルと未訪問エリア内の各特徴ベクトルの類似度が0.125以上かつ,類似度が最も高い既訪問スポットと未訪問スポットの関連付けを行う.
A correlation between the already visited spot having similarity of each feature vector of the already visited spot and each feature vector in the unvisited area of 0.125 or more and the highest similarity and the unvisited spot is performed.

%%%%%%%%%%%%%%%%%%%%%%%%%%%%%%%%%%%%%%%%%%
\subsection{Feature vector generation by TFIDF}
\label{subsec:Feature vector generation by TFIDF}
% 観光スポットのレビューはすべて形態素解析器「mecab-ipadic-NEologd」を使用することで,単語抽出処理を行う.
All tourist spots review words by using morphological analyzer "mecab-ipadic-NEologd".
% しかし,これらを用いて得られた単語は,日本語として成立しない語が含まれており,これらノイズの削除が必要となる.
However, words obtained by using these words contain words that do not hold Japanese, and it is necessary to delete these noises.
% 具体的には,助詞,助動詞,連体詞,記号を削除する.
Specifically, delete particles, auxiliary verbs, rentaishi, symbols.

% 節\ref{subsec:Calculation of similarity by cosine similarity}で関連付けした既訪問スポットと未訪問スポットの類推情報は単語形式でユーザに提示するため,ある観光スポットのレビュー集合を文書$i$とし,$i$に対する単語$j$が出現するスポット集合の出現回数を$TF_{i,j}$,単語$j$がスポット集合の文書数を$DF_{j}$,スポット集合内の全スポット数を$|D|$とした時,そのスポットにおける単語の特徴量は,式\ref{math:TFIDF}で定義される.
Since the analogy information of the visited spot and the unvisited spot associated in section \ref{subsec:Calculation of similarity by cosine similarity} is presented to the user in word format, a review set of a certain tourist spot is assumed to be a document $i$ and a spot where the word $j$ for $i$ appears When the number of occurrences of the set is $TF_{i,j}$, the word $j$ is the number of documents in the spot set is $DF_{j}$, and the total number of spots in the spot set is $|D|$ The feature quantity of a word in a spot is defined by the formula \ref{math:TFIDF}.

\begin{equation}
  word_{i,j} = TF_{i,j} \times IDF_{j}
  \label{math:TFIDF}
\end{equation}
\begin{equation}
  IDF_{j} = log(\frac{|D|}{DF_{j}})
  \label{math:IDF}
\end{equation}

% 本手法では,既訪問スポットに関して,ユーザが複数個のスポットを入力する.
In this method, for visited spots, the user inputs plurality of spots.
% それぞれのスポットの全レビューをまとめて1つの文書と見なし,それ以外のスポットの全レビューも文書とみなすことで,式\ref{math:TFIDF},\ref{math:IDF}によってTFIDF値を算出し,既訪問スポット毎の特徴ベクトルとする.
By considering all reviews of each spot as one document at a time and by considering all reviews of the other spots as documents, the TFIDF value is calculated by the formula \ref{math:TFIDF}, \ref{math:IDF}, and use it as the feature vector for each spot in the unvisited area.

% 未訪問エリアに関して,ユーザがエリアを指定して入力する.
Regarding the unvisited area, the user designates an area and inputs it.
% エリア内のそれぞれのスポットの全レビューをまとめて1つの文書と見なし,それ以外のスポットの全レビューも文書とみなすことで,式\ref{math:TFIDF},\ref{math:IDF}によってTFIDF値を算出し,未訪問エリアのスポット毎の特徴ベクトルとする.
By considering all reviews of each spot in the area as one document and considering all reviews of the other spots as a document, the TFIDF value is calculated by the formula \ref{math:TFIDF}, \ref{math:IDF}, and use it as the feature vector for each spot in the unvisited area.

%%%%%%%%%%%%%%%%%%%%%%%%%%%%%%%%%%%%%%%%%%
\subsection{Presenting similar information by harmonic mean}
\label{subsec:Presenting similar information by harmonic mean}
% 既訪問スポットから未訪問スポットをイメージするための類推情報は,単語形式でユーザに提示する.
The analogy information for imaging the unvisited spot from the already visited spot is presented to the user in word format.
% 節\ref{subsec:Feature vector generation by TFIDF}で関連付けした既訪問スポットと未訪問スポットの類推情報は,節\ref{subsec:Calculation of similarity by cosine similarity}で求めた各スポットの特徴ベクトルによる調和平均を用いて決定する.
The analogy information of the visited spot and the unvisited spot associated with the section \ref{subsec:Feature vector generation by TFIDF}, it is determined using harmonic mean by the feature vector of each spot obtained by the section \ref{subsec:Calculation of similarity by cosine similarity}.
% 調和平均とは,逆数の算術平均の逆数である.
The harmonic mean is the reciprocal of the arithmetic mean of the reciprocal.
% 既訪問スポットのレビュー文書と,未訪問スポットのレビュー文書に,共通して出現する単語を抽出する.
Extracts commonly appearing words in the review document of the already visited spot and the review document of the unvisited spot.
% 抽出した単語のスコアは式\ref{math:Harmonic Mean}によって定義する.
The score of the extracted word is defined by the formula \ref{math:Harmonic Mean}.
% $word_{visited}$と$word_{unvisited}$は同じ単語がそれぞれ既訪問スポットのTFIDF値と未訪問スポットのTFIDF値を示している.
$word_{visited}$ and $word_{unvisited}$ indicate the TFIDF value of the visited spot and the TFIDF value of the unvisited spot, respectively.
% 単語スコアの値が大きと既訪問スポットと未訪問スポットのそれぞれのTFIDF値が大きい,つまり単語がそれぞれの文書に置いて重要度が高いことを示している.
When the value of the word score is large, the TFIDF value of each of the visited spot and the unvisited spot is large, that is, the word has high importance in each document.
% よって,単語スコアの上位10個の単語を類推情報としてユーザに提示する.
Therefore, the top ten words of the word score are presented to the user as analogy information.

\begin{eqnarray}
  score=\frac{1}{\frac{1}{2}(\frac{1}{word_{visited}}+\frac{1}{word_{unvisited}})}
  \label{math:Harmonic Mean}
\end{eqnarray}

%%%%%%%%%%%%%%%%%%%%%%%%%%%%%%%%%%%%%%%%%%
%%%%%%%%%%%%%%%%%%%%%%%%%%%%%%%%%%%%%%%%%%
\section{Evaluation Experiment}
\label{sec:Evaluation Experiment}
%%%%%%%%%%%%%%%%%%%%%%%%%%%%%%%%%%%%%%%%%%
%%%%%%%%%%%%%%%%%%%%%%%%%%%%%%%%%%%%%%%%%%
\subsection{Experiment of feature word representing relation}
\label{subsec:Experiment of feature word representing relation}



\subsection{Comparative experiment of correspondence}
\label{subsec:Comparative experiment of correspondence}



%%%%%%%%%%%%%%%%%%%%%%%%%%%%%%%%%%%%%%%%%%
%%%%%%%%%%%%%%%%%%%%%%%%%%%%%%%%%%%%%%%%%%
\section{Conclusions and Future Work}
\label{sec:Conclusions and Future Work}
%%%%%%%%%%%%%%%%%%%%%%%%%%%%%%%%%%%%%%%%%%
%%%%%%%%%%%%%%%%%%%%%%%%%%%%%%%%%%%%%%%%%%


% use section* for acknowledgement
\section*{Acknowledgment}


The authors would like to thank...

The preferred spelling of the word "acknowledgment" in American
English is without an "e" after the "g." Use the singular heading
even if you have many acknowledgments. Avoid expressions such as
"One of us (S.B.A.) would like to thank ... ." Instead, write "F. A.
Author thanks ... ." Sponsor and financial support acknowledgments
are placed in the unnumbered footnote on the first page, not here.


% Can use something like this to put references on a page
% by themselves when using endfloat and the captionsoff option.
\ifCLASSOPTIONcaptionsoff
  \newpage
\fi



% trigger a \newpage just before the given reference
% number - used to balance the columns on the last page
% adjust value as needed - may need to be readjusted if
% the document is modified later
%\IAENGtriggeratref{8}
% The "triggered" command can be changed if desired:
%\IAENGtriggercmd{\enlargethispage{-5in}}

% references section

% can use a bibliography generated by BibTeX as a .bbl file
% BibTeX documentation can be easily obtained at:
% http://www.ctan.org/tex-archive/biblio/bibtex/contrib/doc/
% The IAENGtran BibTeX style support page is at:
% http://www.michaelshell.org/tex/IAENGtran/bibtex/
%\bibliographystyle{IAENGtran}
% argument is your BibTeX string definitions and bibliography database(s)
%\bibliography{IAENGabrv,../bib/paper}
%
% <OR> manually copy in the resultant .bbl file
% set second argument of \begin to the number of references
% (used to reserve space for the reference number labels box)
\begin{thebibliography}{1}

  \bibitem{Codd01}
    廣嶋 伸章,安田 宜仁,藤田 尚樹,片岡 良治,
      地理情報検索におけるクエリ入力支援のための特徴語の提示,
      第26回人工知能学会全国大会, Vol.26, 1C1-R-5-6, 2012
  \bibitem{Codd02}
    松本 敦志,杉本 徹,
      クチコミから抽出した特徴語を利用する観光地検索支援,
      第75回全国大会講演論文集, Vol.2013, No.1, pp.307-308, 2013
  \bibitem{Codd03}
    上原 尚,嶋田 和孝,遠藤 勉,
      Web上に混在する観光情報を活用した観光地推薦システム,
      社団法人 電子情報通信学会,信学技報, Vol.112, No.367, pp.13-18, 2012
  \bibitem{Codd04}
    野守 耕爾,神津 友武,
      口コミデータにPLSAを適用した観光客目線による観光地分析,
      第29回人工知能学会全国大会, Vol.29, 1J2-OS-18a-2, pp.1-4, 2015
  \bibitem{Codd05}
    Gick, M.L. and Holyoak, K.J.:
    Analogical Problem Solving,
    Cognitive Psychology, Vol.12, pp.306–355, 1980
  \bibitem{Codd06}
    石田 純太,砂山 渡,
      新知識を理解するための類推能力の育成,
      第30回人工知能学会全国大会, Vol.30, 3F3-3, 2016
  \bibitem{Codd07}
    % 死
    砂山 渡, 石田 純太,川本 佳代,西原 陽子,
      類推による説明スキルの獲得支援システム,
      情報処理学会論文誌, Vol.59, No.10, 1922–1931, 2018
  \bibitem{Codd08}
    % 中村 潤,大澤 幸生,
    Jun NAKAMURA, Yukio OHSAWA,
      % 概念創造のための類推思考プロセスにおける迷いの効果,
      Effect of Vacillation on Analogical Thought Process for Concept Creation,
      % 横幹, Vol.2, No.1, p.40-48, 2008
      Oukan, Vol.2, No.1, p.40-48, 2008
  \bibitem{Codd09}
    Gentner, D.: Structure-Mapping:
    A Theoretical Framework for Analogy,
    Cognitive Science, Vol.7, pp.155–170, 1983
  \bibitem{Codd13}
    杉山 将,
      確率分布間の距離推定:機械学習分野における最新動向,
      日本応用数理学会論文誌,Vol.23, No.3, pp.439-452, 2013

\end{thebibliography}


\end{document}
